\PassOptionsToPackage{unicode=true}{hyperref} % options for packages loaded elsewhere
\PassOptionsToPackage{hyphens}{url}
%
\documentclass[]{article}
\usepackage{lmodern}
\usepackage{amssymb,amsmath}
\usepackage{ifxetex,ifluatex}
\usepackage{fixltx2e} % provides \textsubscript
\ifnum 0\ifxetex 1\fi\ifluatex 1\fi=0 % if pdftex
  \usepackage[T1]{fontenc}
  \usepackage[utf8]{inputenc}
  \usepackage{textcomp} % provides euro and other symbols
\else % if luatex or xelatex
  \usepackage{unicode-math}
  \defaultfontfeatures{Ligatures=TeX,Scale=MatchLowercase}
\fi
% use upquote if available, for straight quotes in verbatim environments
\IfFileExists{upquote.sty}{\usepackage{upquote}}{}
% use microtype if available
\IfFileExists{microtype.sty}{%
\usepackage[]{microtype}
\UseMicrotypeSet[protrusion]{basicmath} % disable protrusion for tt fonts
}{}
\IfFileExists{parskip.sty}{%
\usepackage{parskip}
}{% else
\setlength{\parindent}{0pt}
\setlength{\parskip}{6pt plus 2pt minus 1pt}
}
\usepackage{hyperref}
\hypersetup{
            pdftitle={UB13 -- Tutorium Mathe A WS19/20},
            pdfauthor={A. Hanke},
            pdfborder={0 0 0},
            breaklinks=true}
\urlstyle{same}  % don't use monospace font for urls
\usepackage[margin=1in]{geometry}
\usepackage{graphicx,grffile}
\makeatletter
\def\maxwidth{\ifdim\Gin@nat@width>\linewidth\linewidth\else\Gin@nat@width\fi}
\def\maxheight{\ifdim\Gin@nat@height>\textheight\textheight\else\Gin@nat@height\fi}
\makeatother
% Scale images if necessary, so that they will not overflow the page
% margins by default, and it is still possible to overwrite the defaults
% using explicit options in \includegraphics[width, height, ...]{}
\setkeys{Gin}{width=\maxwidth,height=\maxheight,keepaspectratio}
\setlength{\emergencystretch}{3em}  % prevent overfull lines
\providecommand{\tightlist}{%
  \setlength{\itemsep}{0pt}\setlength{\parskip}{0pt}}
\setcounter{secnumdepth}{0}
% Redefines (sub)paragraphs to behave more like sections
\ifx\paragraph\undefined\else
\let\oldparagraph\paragraph
\renewcommand{\paragraph}[1]{\oldparagraph{#1}\mbox{}}
\fi
\ifx\subparagraph\undefined\else
\let\oldsubparagraph\subparagraph
\renewcommand{\subparagraph}[1]{\oldsubparagraph{#1}\mbox{}}
\fi

% set default figure placement to htbp
\makeatletter
\def\fps@figure{htbp}
\makeatother


\title{UB13 -- Tutorium Mathe A WS19/20}
\author{A. Hanke}
\date{Tutorium: 03.02.2020}

\begin{document}
\maketitle

\hypertarget{aufgabe-1}{%
\section{Aufgabe 1}\label{aufgabe-1}}

\begin{itemize}
\tightlist
\item
  a: \(a_nx^n = \sum_{n=0}^{\infty} x^{2n+1}\) mit \(a_n = 1\).\\
  Konvergenzradius:\[ r = lim_{n \rightarrow \infty}\left| \frac{x^{2n + 2} }{x^{2n+1}} \right| = 1\]
  Konvergenz radius \(=1\)
\item
  b: \(a_nx^n = \sum_{n = 0}^{\infty} (-1)^{n} 2^n x^n\)\\
  Konvergenzradius:\[ r = lim_{n \rightarrow \infty}
  \left| \frac{2^{n+1}x^{n+1}}{2^n x^n} \right| = |2x|\] Also gilt
  konvergenz bei: \(|x|< 0.5\) und \(r = 0.5\).
\item
  c: \(P(x) = \sum_{n =1}^{\infty} n x^n\)\\
  Konvergenzradius:\[r = \lim_{n \rightarrow \infty} \left| \frac{(n+1)x^{n+1}}{nx^n} \right| = \frac{n+1}{n} |x| <1 \Rightarrow r = 1\]
\item
  d: \(P(x) = \sum_{n = 1}^\infty \frac{x^n}{n^2}\)\\
  Konvergenzradius:\[r = \lim_{n\rightarrow \infty}
  \left| \frac{\frac{x^{n+1}}{(n+1)^2}}{\frac{x^n}{n^2}} \right| =
  \left|\frac{n^2 x^{n+1}}{(n+1)^2 x^n}\right|=
  \lim \left| \frac{n^2}{(n+1)^2} \right| |x| <1\rightarrow r = 1\]
\end{itemize}

\hypertarget{aufgabe-2}{%
\section{Aufgabe 2}\label{aufgabe-2}}

\hypertarget{a}{%
\subsubsection{(a)}\label{a}}

Reihe:
\[\frac{2}{3} + \frac{2}{3^2} + \frac{2}{3^3} + \dots = \sum_{n = 1}^\infty \frac{2}{3^n} = 2 \sum_{n = 1}^\infty \frac{1}{3^n}\]

Grenzwert: \[
\begin{aligned}
    2 \sum_{n=1}^\infty \frac{1}{3^n} =& 2\underbrace{\sum_{n=0}^\infty \left( \frac{1}{3} \right)^n}_{
    |x|<1 \Rightarrow \sum_{n=0}^\infty x^n = \frac{1}{1-x}
    }- \left( \frac{1}{3} \right)^{n=0}\\
    =&2\frac{1}{1- \frac{1}{3}} -1\\
    =& 1
\end{aligned}
\]

\hypertarget{b}{%
\subsubsection{(b)}\label{b}}

\[
    0.1+\frac{1}{2 !}(0.01)+\frac{1}{3 !}(0.001)+\ldots= \sum_{n=1}^\infty \frac{1}{n!} 0.1^n
\] Erkennen, das es sich um eine Taylor/Mac-Laurin Entwickluong handelt.
\[
    f(x) = \sum_{n=0}^\infty \frac{f^n(0)}{n!} x^n
\] Daraus folgt: \[
    \sum_{n=1}^\infty \frac{1}{n!} 0.1^n = \sum_{n=0}^\infty \frac{1}{n!} 0.1^n - 1 = e^{0.1} -1
\]

\hypertarget{aufgabe-3}{%
\section{Aufgabe 3}\label{aufgabe-3}}

\hypertarget{a-1}{%
\subsubsection{(a)}\label{a-1}}

\[f(x) = \frac{1}{\sqrt{x+1}} = \left( x+1 \right)^{\frac{-1}{2}}\quad \quad x = 0\]
Zunächst ableitungen bestimmen: \[
\begin{array}{ccc}
    f'(x) =& - \frac{1}{2} (x+1)^{-\frac{3}{2}} & f'(0) = -\frac{1}{2}\\
    f''(x) =& - \frac{1}{2} \cdot -\frac{3}{2}(x+1)^{-\frac{5}{2}} & f''(0) = \frac{3}{4}\\
    f'''(x) =& - \frac{1}{2} \cdot -\frac{3}{2} \cdot -\frac{5}{2}(x+1)^{-\frac{7}{2}} & f'''(0) = -\frac{15}{8}
\end{array}
\] Da \(x_0 = 0\) gilt nun für die Taylor Näherung: \[
    f(x) = \sum_{k=0}^3 \frac{f^k(0)}{k!} x^k
\] Und somit hier: \[
    f(x) = 1 - \frac{1}{2}x + \frac{3}{8} x^2 - \frac{5}{16} x^3
\]

\hypertarget{b-1}{%
\subsubsection{(b)}\label{b-1}}

Bestimmen zunächst ableitungen um Reihe zu erkennen: \[
\begin{array}{rl}
    f(x)&=(1-4 x)^{-1} \\ f^{\prime}(x)&=-4 \cdot -1(1-4 x)^{-2} \\
    f^{\prime \prime}(x)&=4 \cdot  -4 \cdot -2(1-4 x)^{-3} \\
    \Rightarrow f^{n}(x)&=(-1)^n(-4)^{n} n!(1-4 x)^{-n-1}=4^{n} n! (1-4 x)^{-n-1}\\
    \Rightarrow f^{n}(0) &= 4^n n!
\end{array}
\]

Konvergenzradius bestimmen (erinnere:
\(f(x)=\left(\sum_{n=0}^{\infty} \frac{f^{(n)}\left(x_{0}\right)}{n !}\left(x-x_{0}\right)^{n}\right)\)):
\[
\begin{align}
    x_0 = 0
    \Rightarrow f(x) &= \sum_{n=0}^\infty 4^n x^n\\
    \text{Wurzelkriterium:}& \lim_{n\rightarrow \infty} 
    \sqrt[n]{|4^nx^n|}\\
    &= \left| 4x \right| < 1\\
    &\Rightarrow \text{Konvergenz für: } |x| < \frac{1}{4}\\
    \Rightarrow & r=\frac{1}{4}
\end{align}
\]

\hypertarget{c}{%
\subsubsection{(c)}\label{c}}

Reihe erkennen: \[
\begin{align}
    f(x)&=e^{1-x} \\
    &= ee^{-x} \\
    = e \sum_{n=0}^\infty \frac{(-x)^n}{n!}
\end{align}
\] Konvergenzradius: \[
\begin{align}
    \text{Quotienten:}& \lim_{n\rightarrow \infty} 
    \left| \frac{\frac{(-x)^{n+1}}{(n+1)!}}{\frac{(-x)^n}{n!}} \right|\\
    &=  \lim_{n\rightarrow \infty} \left| \frac{n!}{(n+1)!} -x \right|\\
    &= 0\\
    \Rightarrow & r=\infty
\end{align}
\]

\hypertarget{aufgabe-4}{%
\section{Aufgabe (4)}\label{aufgabe-4}}

\begin{itemize}
\tightlist
\item
  a: \(a = 2\)\\
  \[
  f(x) \sim a^4 + 4a^3 (x - a) = 16.032
  \] Fehler: \(\ensuremath{2.4008001\times 10^{-5}}\)
\item
  b: \(a = 0\) \[
  f(x) \sim \sin(a) + \cos(a) (x - a) = 0.02
  \] Fehler: \(\ensuremath{-1.3333067\times 10^{-6}}\)
\item
  c: \(a = 0\) \[
  f(x) \sim \cos(a) + \sin(a) (x - a) = 1
  \] Fehler: \(\ensuremath{-4.4996625\times 10^{-4}}\)
\item
  d: \(a = 16\) \[
  f(x) \sim  a^{\frac{1}{4}} + \frac{1}{4}a^{-\frac{3}{4}} (x - a) = 1.9996875
  \] Fehler: \(\ensuremath{-7.3268902\times 10^{-8}}\)
\item
  e: \(a = 1\) \[
  f(x) \sim a^{-1}  - a^{-2} (x - a) = 1.02
  \] Fehler: \(\ensuremath{4.0816327\times 10^{-4}}\)
\item
  f: \(a = \pi\) \[
  f(x) \sim \sin(a) + \cos(a) (x - a) = 0.0015927
  \] Fehler: \(\ensuremath{-6.7330629\times 10^{-10}}\)
\end{itemize}

\end{document}
